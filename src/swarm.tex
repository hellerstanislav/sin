
\documentclass[a4paper,12pt]{article}
\usepackage[czech]{babel}
\usepackage[pdftex]{graphicx}
\usepackage[utf8]{inputenc}

\usepackage{wrapfig}
\usepackage{verbatim} 
\usepackage[unicode]{hyperref}

\usepackage[font=small,labelfont=bf,tableposition=top]{caption}
\DeclareCaptionLabelFormat{andtable}{#1~#2  \&  \tablename~\thetable}

\newcommand{\HRule}{\rule{\linewidth}{0.2mm}}

\let\stdsection\section
\renewcommand\section{\newpage\stdsection}

\begin{document}

% Uvodni strana

\begin{titlepage}

\begin{center}


FIT VUT v Brně, 2012\\[1.5cm]
\includegraphics[width=7cm]{./img/fit-cz.png}\\[1cm]

% Title
\HRule \\[0.4cm]
{\huge \bfseries Swarm Intelligence}\\[0.1cm]
{\LARGE Inteligence roje}
\HRule \\[0.8cm]

{\large Učební text k předmětu Inteligentní systémy (SIN)}\\[1cm]

\vfill

\end{center}

\begin{minipage}{0.4\textwidth}
\begin{flushleft} \large
\emph{Autoři textu:}\\
Bc. Tomáš \textsc{Kimer}\\
Bc. Stanislav \textsc{Heller}
\end{flushleft}
\end{minipage}

\end{titlepage}


% Obsah
\tableofcontents

% Text
\section{Úvod}
Pro tuto kapitolu se bude výborně čerpat z \cite{Blum08SwarmInt} \\
\url{http://www.springerlink.com/content/978-3-540-74089-6#section=226923&page=7&locus=23}

\begin{itemize}
  \item Obecné kecy o tom, k čemu to je, čím je to inspirováno atp.
  \item Zasazení do rámce s agenty a roboty
  \item blabla
\end{itemize}

\subsection{Biologický základ inteligence roje}
Čerpat z \cite{Blum08SwarmInt}, kapitola Biological Foundations of Swarm Intelligence
a z \cite{fleischer2005}, kapitola II/a Observations of Social Insects.
\begin{itemize}
  \item pozorování sociálního hmyzu (mravenci, včely, atp.)
  \item mechanismy pro řešení problémů
  \item feromony, vypařování
  \item komunikace a interakce mezi hmyzem (resp. agenty)
\end{itemize}


\subsection{Možnosti využití}
Kecy o tom, jak je možné swarm využít v:
\begin{itemize}
  \item telekomunikační systémy (optimalizace)
  \item robotika - výrobní, provozní a inspektční systémy, zemědělství, medicína
  \item logistika, kooperativní transport
  \item vojenství
\end{itemize}



\section{Definice základních pojmů}
\begin{itemize}
  \item Samoorganizace (\cite{fleischer2005})
  \item Emergentní jev (\cite{fleischer2005})
  \item Kolektivní inteligence (\cite{fleischer2005})
  \item Schopnost adaptace
\end{itemize}


\section{Inteligence roje v optimalizaci}
Obecné kecy o optimalizaci...

\subsection{Optimalizace mravenčí kolonií}
Pro algoritmy typu ACO bude vhodné využít \cite{Dorigo06antcolony}

\subsection{Optimalizace hejnem částic}
Tady se bude brát hlavně z \cite{Blum08SwarmOpt} a také třeba
z \cite{Blum08SwarmInt}, kapitola Swarm Intelligence in Optimization.



\section{Hejna robotů}
Oblast, která je v odborné literatuře označována jako {\it Swarm Robotics} je jednou z...blabla.

Tady se bude čerpat z \cite{Dudek93}. Určitě bude nutné najít další zdroje.



\section{Další přístupy a metody}
\begin{itemize}
  \item cellular robotic systems, učení a adaptační strategie
  \item chaos \& swarm
\end{itemize}



\renewcommand{\refname}{\section{Použitá literatura}}
\bibliographystyle{plain}
\bibliography{refs}

\end{document}
