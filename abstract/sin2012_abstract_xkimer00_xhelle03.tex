
\documentclass[a4paper,12pt]{article}
\usepackage[czech]{babel}
\usepackage[pdftex]{graphicx}
\usepackage[utf8]{inputenc}
\usepackage{fancyvrb}
\usepackage{wrapfig}
\usepackage{verbatim}
\usepackage{hyperref}

\setlength{\hoffset}{-1.7cm} 
\setlength{\voffset}{-3cm}
\setlength{\textheight}{26.3cm} 
\setlength{\textwidth}{16.7cm}

\DefineShortVerb{\|}

\usepackage[font=small,labelfont=bf,tableposition=top]{caption}
\DeclareCaptionLabelFormat{andtable}{#1~#2  \&  \tablename~\thetable}

\title{Projekt do SIN 2012 - abstrakt}
\author{Tomáš Kimer xkimer00, Stanislav Heller xhelle03}

\begin{document}
\maketitle

\section*{Zadání č. 7 - podklady pro výuku}
\vspace{5mm}
\subsection*{Text zadání}
Podklady pro výuku. Podklady pro výuku (učební text s odkazy na zdroje + prezentace cca 30 snímků v openoffice + otázky ke zkoušce) - nesmí se to tématicky významně překrývat žádným předmětem, vyučovaným na FIT. Zpracováno může být libovolné téma, které zapadá do rámce předmětu SIN a není předáškami SIN v potřebné míře pokryto. Pro základní orientaci prostudujte podklady k souvisejícím přednáškám (i když ještě nebyly odpřednášeny). (Doporučená velikost týmu: 1-2 řešitelé)

\subsection*{Téma: Inteligence roje (Swarm Intelligence)}

\subsection*{Zdoje informací}
Souhrn předpokládané literatury, ze které hodláme čerpat informace pro vytvoření studijních materiálů a prezentace.
\begin{itemize}
  \item {\bf Swarm Intelligence: Literature Overview} 
        \url{http://pdf.aminer.org/000/352/063/optimization\_of\_group\_behavior\_on\_cellular\_robotic\_system\_in\_dynamic.pdf}
  \item {\bf Foundations of Swarm Intelligence (2003)}
        \url{http://arxiv.org/abs/nlin/0502003} 
  \item {\bf Swarm Intelligence in Optimization}
        \url{http://citeseerx.ist.psu.edu/viewdoc/summary?doi=10.1.1.156.1427}
  \item {\bf Chaos and Swarm Intelligence}
        \url{http://citeseerx.ist.psu.edu/viewdoc/summary?doi=10.1.1.161.3773}
  \item {\bf The Ant System: Optimization by a colony of cooperating agents (1996)}
        \url{http://citeseerx.ist.psu.edu/viewdoc/summary?doi=10.1.1.26.1865}
  \item {\bf Ant Colony Optimization – Artificial Ants as a Computational Intelligence Technique (2006)}
        \url{http://citeseerx.ist.psu.edu/viewdoc/summary?doi=10.1.1.70.1052}
  \item {\bf A Taxonomy for Swarm Robots}
        \url{http://ieeexplore.ieee.org/stamp/stamp.jsp?tp=&arnumber=583135}
\end{itemize}


\subsection*{Programové vybavení}
\LaTeX{}, pro prezentaci Beamer, resp. pokud bude vyžadováno, tak OpenOffice. Otázky ke zkoušce v samostatném dokumentu v \LaTeX{}u.

\end{document}

